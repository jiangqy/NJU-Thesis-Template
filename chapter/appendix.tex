\chapter{符号及简称说明}\label{chp:appendix}
\section{本文使用的符号}
表\ref{tab:appendix-notaions}中列出了一些本文使用的通用符号。
\begin{table}[H]
\centering
\caption{本文使用的符号}
\label{tab:appendix-notaions}
\begin{tabular}{ll}
\toprule
符号                    &    说明                                    \\\midrule
$\varnothing$          &    空集                                    \\
$\RB$                  &    实空间                                   \\
$\RB^c$                &    $c$维实空间                              \\ 
$\{-1,+1\}^c$          &    $c$维二值空间                            \\
$\B$                   &    大写黑体字母,矩阵                        \\
$\B_{i*}$              &    矩阵$\B$的第$i$行                        \\
$\B_{*j}$              &    矩阵$\B$的第$j$列                        \\
$B_{ij}$               &    矩阵$\B$中行标为$i$,列标为$j$的元素        \\
$\b$                   &    小写黑体字母,列向量                       \\
$b_i$                  &    向量$\b_i$的第$i$个元素                   \\
$\I_n$                 &    维度为$n\times n$的单位矩阵               \\
$\bone_d$              &    元素全部为1的$d$维向量                    \\
$\bone_{n\times n}$    &    元素全部为1的$n\times n$的矩阵            \\
$\bzero_d$             &    元素全部为0的$d$维向量                    \\
$\bzero_{n\times n}$   &    元素全部为0的$n\times n$的矩阵            \\
$\B^\top$              &    矩阵$\B$的转置                           \\
$\B\succeq\bzero$      &    半正定矩阵                               \\
$\A\succeq\B$          &    矩阵$\A-\B$为半正定矩阵                   \\
$\B\succ\bzero$        &    正定矩阵                                \\
$\A\succ\B$            &    矩阵$\A-\B$为正定矩阵                    \\
\bottomrule
\end{tabular}
\end{table}

此外,一些函数定义如下。
\begin{itemize}
\item 矩阵的迹:
\begin{align}
\tr(\B)=\sum_{i=1}^nB_{ii}.\label{fun:appendix-trace}
\end{align}
\item 向量的2范数:
\begin{align}
\Vert\b\Vert_2=\sqrt{\sum_{i=1}^nb_i^2}.\label{fun:appendix-2norm}
\end{align}
\item 矩阵的Frobenius范数:
\begin{align}
\Vert\B\Vert_F=\sqrt{\sum_{i,j=1}^nB_{ij}^2}.\label{fun:appendix-Fnorm}
\end{align}
\item 矩阵的1范数:
\begin{align}
\Vert\B\Vert_1=\sum_{i,j=1}^n\vert B_{ij}\vert.\label{fun:appendix-1norm}
\end{align}
\item 指示函数:
\begin{align}
\hone(condition)=\left\{
\begin{array}{ll}
1   & \text{if } condition \text{ is true},\\
0   & \text{otherwise.}
\end{array}
\right.\label{fun:appendix-indicator}
\end{align}
\end{itemize}

\section{简称说明}
表\ref{tab:appendix-abbr}列出了一些本文使用的简称的含义。
\begin{longtable}{ll}
\caption{本文使用的通用简称}
\label{tab:appendix-abbr}
\endfirsthead
\endhead
\toprule
简称         & 全称                                                   \\\midrule
BQP         & Binary Quadratic Programming,离散二次规划               \\
PQ          & Product Quantization,乘积量化                          \\
\bottomrule
\end{longtable}

